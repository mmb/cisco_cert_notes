\documentclass{article}

\begin{document}

\section{Collision Detection}

\begin{description}

\item[CSMA/CD]
carrier sense multiple access with collision detection

\end{description}

Shared bus\\

Collision domains

\begin{enumerate}

\item Listen to wire
\item If clear, send
\item If collision detected, send jam signal, wait random backoff timer, goto 1

\end{enumerate}

\section{Cable}

Standard ethernet cable is CAT5 UTP (category 5 unshielded twisted pair).
Twisting cables cuts down on interference.\\

Ethernet is logical bus (as opposed to ring).\\

\begin{tabular}{ | c | c | c | c | c | }
\hline
Name & Speed & Cable type & Standards & Max cable length \\ \hline
10BaseT & 10 Mbps & Twisted pair & IEEE 802.3 & 100m \\ \hline
10Base5 & & Coax & & 500m \\ \hline
10Base2 & & Coax & & 185m \\ \hline
Fast ethernet & 100 Mbps & & IEEE 802.3u & 100m \\ \hline
Gigabit & 1000 Mbps & not copper & IEEE 802.3z, 802.ab & 100m \\ \hline
\end{tabular}

\section{Pins}

Pins are wire endpoints. One set of pins is used to transmit and one pair
is used to receive.\\

Pins 1 and 2 transmit.\\

Pins 3 and 6 receive.

\section{Crosstalk}

Crosstalk is when signal crosses over from one cable to another. It is highest
where the data enters the cable.

\begin{description}

\item[NEXT]
near end crosstalk. ``Near end'' is a relative term referring to the end being
tested. Often caused by crossed or crushed wire.

\item[FEXT]
far end crosstalk

\item[PSNEXT]
power sum near end crosstalk, the calculation used for the NEXT test

\end{description}

\section{Cable Types}

\begin{description}

\item[straight through cable]
used to connect a normal PC to a switch or hub, pin 1 connects to pin 1, pin 2
to pin 2

\item[crossover cable]
used to connect two similar pieces of networking gear (such as two switches
(trunk)), pin 1 connects to pin 3, pin 2 to pin 6

\item[rollover cable]
used to connect to the console port on a switch or router for configuration and
debugging, all eight wires roll over, pin 1 connects to pin 8, pin 2 to pin 7,
etc.

\end{description}

\section{MAC addressing}

Media access control, AKA physical address, BIA (burned in address), NIC
address, LAN address, ethernet address.\\

48 bits\\

First half (24 bits) is the OUI which is the vendor id.

\begin {description}

\item[unicast]
to one host

\item[multicast]
to some hosts (\texttt{01:00:5e:00:00:00} - \texttt{01:00:5e:7f:ff:ff})

\item[broadcast]
to all hosts (\texttt{ff:ff:ff:ff:ff:ff})

\end{description}

\section{WAN cabling}

Cisco routers use serial cables for serial interfaces (such as frame relay).

DTE/DCE connects two Cisco routers by serial cable.

\end{document}
