\section{Introduction to Networking and the Networking Models}

\subsection{OSI Model}

\begin{tabular}{ | c | c | }
\hline
7 & application \\ \hline
6 & presentation \\ \hline
5 & session \\ \hline
4 & transport \\ \hline
3 & network \\ \hline
2 & data link \\ \hline
1 & physical \\ \hline
\end{tabular}

\begin{description}

\item[application]
end user, authentication, ensure remote communication partner is available,
data integrity, privacy and error recovery, firewalls, email, telnet, HTTP,
FTP, SNMP

\item[presentation]
how should data be presented, formatting, encryption, compatibility with OS,
encapsulation for transmission, data formatting, data encryption, compression,
translation

\item[session]
manager of communication between hosts, creation, maintenance and teardown

\item[transport]
establish logical end-to-end connection between two systems, segment data,
correct order, free of errors, TCP (connection), UDP (connectionless)

\item[network]
IP, routing layer, what valid paths to destination?, what is the best path?

\item[data link]
switches, WAPs, ethernet, HDLC (high data link control), PPP (point-to-point
protocol), frame relay, error detection (frame check sequence) but not error
recovery, MAC addresses, bridges

\item[physical]
pins, connectors, current, hubs

\end{description}

Layer 3 switches exist.\\

Data flows down the model, then back up.

\subsubsection{Data units}

\begin{tabular}{ | c | c | }
\hline
Layer & Unit \\ \hline \hline
7 - 5 & data \\ \hline
4 & segment \\ \hline
3 & packet \\ \hline
2 & frame \\ \hline
1 & bit \\ \hline
\end{tabular}

Each layer adds a layer-specific header that is removed by the same layer at
the other end (same-layer interaction). Layer 2 adds a trailer and a header.

\begin{description}

\item[PDU]
protocol data unit, combination of header and data

\item[adjacent-layer interaction]
interaction between adjacent layers on the same host

\end{description}

\subsection{TCP/IP Model}

\begin{tabular}{ | c | c | }
\hline
TCP/IP Model Layer & OSI equivalent \\ \hline \hline
application & application, presentation, session \\ \hline
transport & transport \\ \hline
internet & network \\ \hline
network access & data link, physical\\ \hline
\end{tabular}

\subsection{TCP}

\begin{itemize}

\item guaranteed delivery
\item error detection via sequence and ack numbers
\item windowing
\item connection-oriented

\end{itemize}

\subsubsection{3-way Handshake}

Done before data can be transmitted. Negotiate TCP sequence number.

\begin{enumerate}

\item Sender sends segment with SYN bit set
\item Receiver sends segment with SYN and ACK bits set (SYN/ACK)
\item Sender sends segment with ACK bit set

\end{enumerate}

TCP sends FIN to tear down.

\subsubsection{SYN / ACK Flow}

TCP performs both error detection and recovery.\\

The sequence number gives order segements should be reassembled.\\

Recipient sends ACK number of next sequence number it expects to see (not the
last sequence number received). This is how it detects lost segments.\\

Sender sets acknowledgement timer. If an ACK is not received by the time
it expires, sender will retransmit all segments in question (PAR,
positive acknowledgement with retransmission).

\subsubsection{Windowing}

Window size is how much data the sender can transmit without an ACK.\\

Recipient decides window size (flow control).\\

Recipient increases window size until error rate goes up, then reduces (sliding
window).

\subsection{UDP}

\begin{itemize}

\item best effort delivery, no guarantee
\item no error detection
\item no windowing
\item connectionless

\end{itemize}

Less overhead than TCP.

\subsection{Ports}

Used for both TCP and UDP.\\

For sending multiple flows of information to the same host (multiplexing).

\begin{description}

\item[socket]
combination of IP address and port

\end{description}

Well known ports are reserved ports less than 1024.\\

\begin{tabular}{ | c | c | }
\hline
FTP & TCP 20, 21 \\ \hline
SSH & TCP 22 \\ \hline
Telnet & TCP 23 \\ \hline
SMTP & TCP 25 \\ \hline
HTTP & TCP 80 \\ \hline
POP3 & TCP 110 \\ \hline
SSL & TCP 443 \\ \hline
\hline
DHCP & UDP 67, 68 \\ \hline
TFTP & UDP 69 \\ \hline
SNMP & UDP 161 \\ \hline
\hline
DNS & TCP and UDP 53 \\ \hline
private email & TCP and UDP 24 \\ \hline
\hline
VOIP & UDP 16384 - 32767 \\ \hline
\end{tabular}
