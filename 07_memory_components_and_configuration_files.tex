\section{Memory Components and Configuration Files}

\subsection{ROM, RAM, NVRAM and Flash}

\begin{description}

\item[ROM]
read-only memory, stores the router's startup bootstrap program, operating
system software, and power-on diagnostic test programs (POST)

\item[Flash Memory]
where IOS images are kept, erasable and reprogrammable ROM, retained by the
router on reload

\item[RAM]
random access memory, stores operational information such as routing tables
and the running configuration file, lost when router is powered down or
reloaded

\item[NVRAM]
non-volatile RAM, holds the router's startup configuration file, not lost
when router is powered down or reloaded
\end{description}

\subsection{Cisco Router Boot Process}

\begin{enumerate}

\item A series of POSTs (power-on self test) are run that verify the basic
operation of network interfaces, memory and CPU. Detects major problems early
in the boot process, such as a broken fan.

\item If the router passes POST, it looks for a source to load an IOS
(internetwork operating system) image from. It will look, in order, in
Flash memory, a TFTP (trivial file transfer protocol) server, and ROM (like
Windows safe mode). This order can be changed in the configuration register.

\item Once the IOS image is found, the router will look for a startup
configuration file. It will look in NVRAM first, and if not found it will
look for a TFTP server with a startup file. If no valid startup
configuration file is found, the router enters setup mode and runs the
system configuration dialog.\\

If boot completes successfully on a Catalyst Switch, the console will show
the > prompt.

\end{enumerate}

Out of the box a Cisco router or switch has no configuration. It must
be configured using either setup mode or the CLI.\\

Control-C can be used to abort setup mode.\\

To enter setup mode from the router prompt use the \texttt{setup} command.\\

\subsection{Configuration Files}

A router has two configuration files at any given time: start-config and
running-config. They usually have the same contents but can differ if you
change the running configuration but do not \texttt{copy run start}.\\

\subsubsection{copy command}

\texttt{copy <from> <to>}

\subsection{Configuration Backups}

Can be useful when:

\begin{itemize}

\item network attackers change or delete the configuration
\item a network engineer makes a mistake
\item the configuration gets corrupted

\end{itemize}

A TFTP server can be used for backups.\\

\subsection{IOS Updates}

There are many different IOS images on Cisco's website and which ones you have
access to depends on your service level.\\

As a new IOS image is loaded exclamation points show progress. If they stop
for an extended period, there is a problem with the copy.\\

After a new IOS has been loaded to Flash, the router needs to be reloaded.\\

\texttt{show flash} shows how much free space is available in Flash.

\subsection{Configuration Register}

\texttt{show version} shows the current configuration register value.\\

The \texttt{config-register} command changes the configuration register value.
Changing the configuration register value requires a reload to take effect.
Using an invalid value can brick the router.\\

Common configuration register settings:

\begin{description}

\item[0x2102]
default, router looks for startup config in NVRAM and for IOS image in Flash

\item[0x2142]
NVRAM contents are bypassed, startup configuration is ignored, used to reset
password

\item[0x2100]
router boots into ROM monitor mode

\end{description}

You have to be physically present at the router to perform password recovery.\\

You must remember to change the configuration register back to 0x2102 after
using another value (such as after a password reset).
