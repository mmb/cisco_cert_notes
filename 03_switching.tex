\section{Switching}

\subsection{Hubs and Repeaters}

\begin{description}

\item[repeater]
generates a new, clean copy of existing signal

\item[attenuation]
weakening of a signal as it travels

\item[hub]
a repeater with multiple ports

\end{description}

Hubs and repeaters are layer 1 devices. They do not perform switching or look
at MAC addresses.\\

PCs connected to a hub share bandwidth and only one can transmit at a time.
Collision detection is done using CSMA/CD.\\

A hub is one big broadcast domain.

\subsection{Bridges}

Bridges are places between multiple hubs. They create smaller collision
domains, but still have only one broadcast domain.

\subsection{Switches}

Each host connected to a switch has its own collision domain.\\

By default, Cisco switches have all ports in the same broadcast domain.

\begin{description}

\item[microsegmentation]
each host is in its own collision domain

\end{description}

\subsubsection{Packet Actions}

When a packet comes into a switch, the switch will check its MAC table.\\

If the source address in the frame is not in the table it will be added
with the port it came in on.\\

The switch will then:\\

\begin{description}

\item[forward]
when the switch has an entry for the destination MAC address in its MAC table,
it will be sent out the corresponding port

\item[flood]
when there is no entry for the destination MAC address in its MAC table,
it will be sent out every port except the one it came in on

\item[filter]
when both the source and destination MAC addresses are found in its MAC table
and are associated with the same port the frame will be dropped

\end{description}

\begin{description}

\item[unknown unicast frame]
a frame coming into a switch when the switch does not have the destination
address in its MAC table, always flooded

\end{description}

Switches never send a frame back out the same port it came in on.\\

The MAC address for broadcast frames is \texttt{ff:ff:ff:ff:ff:ff}.\\

The MAC table is also known as the CAM (content addressable memory) table,
bridging table or switching table.\\

The default age-out time for dynamic MAC table entries is 300 seconds.\\

VLAN 1 is the default VLAN on Cisco switches.

\subsubsection{Forwarding Methods}

When a switch decides to forward a frame, it will either:

\begin{description}

\item[store-and-forward]
the entire frame is stored by the switch, checked for errors (error detection,
not correction), and then forwarded, best error detection of all the methods

\item[cut through]
frame is forwarded immediately without checking FCS, can be forwarded before
the entire frame is received, fastest of all the methods

\item[fragment-free]
checks first 64 bytes of of frame for errors, compromise


\end{description}

FCS lets recipient determine if frame is corrupted. It provides error
detection but not correction.

\subsection{VLANs}

\begin{description}

\item[broadcast storm]
continual generation of new broadcasts, can kill switch by overwhelming
CPU and memory

\end{description}

VLANs are a way to separate groups of ports and can be used to limit broadcast
propagation.\\

No traffic can be sent from one VLAN to another without the intervention of
a layer 3 device (such as a router).

\subsection{Switch Types}

The Cisco 3-layer switching model.

\begin{description}

\item[access]
switches that end users are connected to

\item[distribution]
intermediate

\item[core]
connect different buildings on a campus network

\end{description}

\subsection{Spanning Tree Protocol}

STP is used to prevent swtiching loops. It is enabled by default on Cisco
routers.\\

STP determines the best path, and if that goes down, will determine a new best
path. Ports that are not on the best path will be put in blocking mode.\\

The physically shortest path might not be the best one.

\subsection{Switch Security}

\begin{itemize}

\item lock up rooms and cabinets
\item close unused ports with shutdown
\item prevent trunking with switch mode access
\item place unused ports in an unused VLAN

\end{itemize}

\subsubsection{Port Security}

Can be used to limit use of ports by MAC address.\\

Options:

\begin{description}

\item[aging]
how long allowed MACs will remain allowed

\item[maximum]
how many secure MACs are allowed (default 1)

\item[violation]
what happens when a non-secure MAC is seen

\begin{description}

\item[shutdown]
default, shut down port, log message, and drop frame, port must be manually
reopened

\item[restrict]
logs and drops frames, but does not shutdown port

\item[protect]
drops frames only

\end{description}

\end{description}

\begin{description}

\item[sticky port security]
the first MAC seen on the port is allowed

\end{description}

When a port is in err-disabled state the port LED turns off and the port
must be manually re-enabled.
