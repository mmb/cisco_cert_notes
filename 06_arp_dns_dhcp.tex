\section{ARP, DNS and DHCP}

Data going from host A to host B must have a destination IP address for host B
and a destination MAC address for host B.

\subsection{DNS process}

Each host has the IP address of at least one DNS server.\\

The DNS server address can be hardcoded on the host or it can be obtained
via DHCP.\\

A host asks its DNS server what the IP address of the destination host is.

\subsection{ARP}

\begin{description}

\item[Address resolution protocol]
allows a host to get a remote device's layer 2 address when it knows the
remote device's layer 3 address

\end{description}

There are no ARP servers.\\

ARP process uses a series of broadcasts and (unicast) replies.\\

If host A needs host B's MAC address:

\begin{enumerate}

\item host A sends out a layer 2 broadcast, the source MAC address will be
host A's MAC, the destination MAC will be ff:ff:ff:ff:ff:ff, source IP address
will be host A's IP, destination IP will be host B's IP address

\item the only device that will reply is the one whose IP address is the
destination IP address of the ARP request, it will unicast a reply to host A
telling host A its MAC address

\item host A now knows host B's IP and MAC, so it can send data to host B

\end{enumerate}

\subsubsection{ARP Caches}

Hosts build their own ARP caches to limit broadcasts.\\

\texttt{arp -a} to see the ARP cache in Windows.\\

Routers also keep ARP caches which can be viewed with the command
\texttt{show arp}.

\subsubsection{Proxy ARP}

Switches don't affect the ARP process (they just forward), but routers do.\\

Routers do not forward broadcasts. Instead they use proxy ARP to answer the
ARP request with the MAC address of the router interface that received the
original request. The requesting host has no idea that the MAC address it
has received is actually the MAC address of a router interface and not the
MAC address of the destination host.\\

To get the actual MAC address of a destination host, the router will consult
its ARP cache or generate an ARP request of its own.

\subsection{DCHP}

For a host to communicate with another host it needs to have the following
configured on it:\\

\begin{itemize}

\item IP address
\item network mask
\item IP addresses of DNS servers
\item default gateway

\end{itemize}

This could all be configured manually on each host, or it could be configured
using DHCP (dynamic host configuration protocol). DHCP is easier to manage.

\subsubsection{DHCP Process}

\begin{enumerate}

\item DHCP client broadcasts DHCP DISCOVER, since it is broadcast it will not
be forwarded by routers

\item any DHCP server can broadcast a DHCP OFFER, which contains the IP
address, network mask, and lease time (amount of time the client is allowed to
use the configuration) that the DHCP server is offering, as well as the IP
address of the DHCP server making the offer

\item client may receive more than one DHCP OFFER and will accept the first
one it receives by sending a DHCP REQUEST, the DHCP REQUEST is broadcast and
when other DHCP servers see that their offer was not accepted they will
return the IP address they offered to their address pool

\item the DHCP server sends the client a DHCP ACKNOWLEDGEMENT with the rest of
the information it needs

\end{enumerate}

\subsubsection{Cisco Routers as DHCP Servers}

Some Cisco routers can act as DHCP servers.\\

Cisco Security Device Manager (SDM) is a GUI tool that can be used to
configure a DHCP server.\\

The \texttt{show flash} command can be used to determine if SDM is loaded on
a router.\\

The \texttt{sh ip dhcp conflicts} command can be used to see if an address
in the DHCP pool is already in use by another device.
