\section{Common Router and Switch Commands}

\subsection{Initial Configuration Problems}

\begin{enumerate}

\item some desired features will not be enabled
\item some undesired features will be enabled

\end{enumerate}

\subsection{Ways to Connect to Cisco Routers and Switches}

\begin{enumerate}

\item physically, using laptop and rollover cable (RJ-45 to DB-9)
\item telnet or ssh

\end{enumerate}

\subsection{Protecting the Console Port with a Password}

\begin{verbatim}
conf t
line con 0
password SECRET
login
^Z
wr
\end{verbatim}

\subsection{Enabling Telnet}

To enable telnet, you must have a password in the VTY lines.

\begin{verbatim}
conf t
line vty 0 4
password SECRET
login
^Z
wr
\end{verbatim}

\texttt{line vty 0 4} means 5 virtual terminal lines\\

When telnetting in, by default you are in user exec mode.\\

To enable a telnet session to receive console output:

\begin{verbatim}
terminal monitor
\end{verbatim}

\subsection{Auxiliary Port}

The preferred method for a network engineer to log in to a remote device is
to use a modem connection to the auxiliary port.

\subsection{Command Modes}

\begin{description}

\item[$>$ in prompt]
user exec mode

\item[\# in prompt]
privileged exec mode (root), to get to this mode use the \texttt{enable}
command, to leave it use the \texttt{disable} command

\end{description}

User exec mode cannot be protected with a password.\\

\texttt{conf t} (configure terminal) enters global configuration mode.\\

\texttt{enable password SECRET} sets the password in clear text (can be seen
in configuration).\\

\texttt{enable secret SECRET} sets the password encrypted (shows hashed in
configuration), takes precedence over enable password, always the preferred
method.\\

\texttt{enable secret 5 XXX} in the configuration means encryption level 5.\\

When users connect via telnet, an enable password is mandatory.\\

No password is required for the console, but it is recommended.\\

The password on VTY lines must be set to allow telnet or ssh.

\subsubsection{Automatically Enabling Privileged Mode}

To automatically enable privileged exec when a user logs in via telnet, set
privilege level to 15:

\begin{verbatim}
conf t
line vty 0 4
privilege level 15
\end{verbatim}

\subsection{Setting up SSH}

When using telnet, nothing is encrypted, including passwords. SSH encrypts
everything.

\begin{verbatim}
conf t
line vty 0 4
no password SECRET
no login
^Z

conf t
line vty 0 4
login local
exit
username test password SECRET
^Z
wr
\end{verbatim}

SSH requires the following to be set in the global configuration:

\begin{verbatim}
ip domain-name
crypto key generate rsa
\end{verbatim}

And on the VTY lines:

\begin{verbatim}
transport input ssh
\end{verbatim}

\subsubsection{Setting User Privilege}

\begin{verbatim}
username test privilege 15 password SECRET
\end{verbatim}

\subsection{LEDs}

\begin{description}

\item[green]
good

\item[amber]
a problem only if it remains on

\item[dark]
bad

\end{description}

\begin{description}

\item[SYST]
system, if it's green the switch is on, if dark, off, amber means
Power On Self Test (POST) failed (broken fan, etc.)

\item[RPS]
redundant power supply, green means ok

\item[STAT]
when green, port status lights are operational

\item[DUPLEX]
green means full duplex, off means half duplex

\item[SPEED]
green is 100 Mbps, dark is 10 Mbps, flashing green is 1000 Mbps

\end{description}

When connecting two switches, and you don't see any lights, you may have used
a straight-through cable instead of a crossover cable.

\subsection{Switch Management Interface}

For users to connect to a switch via telnet or SSH, it must have an IP address
assigned to vlan 1 (the switch management interface).

\begin{verbatim}
interface vlan1
ip address 192.168.0.2 255.255.255.0
no shutdown
show interface vlan 1
\end{verbatim}

\subsection{Assigning a Default Gateway to a Switch}

\begin{verbatim}
ip default-gateway 192.168.0.1
\end{verbatim}

\subsection{Interface Range}

To apply the same command to a range of interfaces at once:

\begin{verbatim}
interface range fast 0/1 - 6
speed 100
\end{verbatim}

\subsection{Banner}

\subsubsection{motd}

motd is shown after connect.

\begin{verbatim}
conf t
banner motd $
This is the message text. $
\end{verbatim}

\subsubsection{login}

login banner is shown before login.

\begin{verbatim}
conf t
banner login $
Hackers log off now. $
\end{verbatim}

\subsubsection{exec}

exec banner is shown after login.

\begin{verbatim}
conf t
banner exec $
Welcome. $
\end{verbatim}

\subsection{logging synchronous}

Hold logging messages until user input and output is done, so log messages
don't display in the middle of user input and output.

\begin{verbatim}
conf t
line con 0
logging synchronous
\end{verbatim}

\subsection{exec-timeout}

exec-timeout sets the session inactivity timer (amount of time before you will
be automatically logged out). The first number is in minutes, the second
is in seconds. 0 0 disables session timeout.

\begin{verbatim}
conf t
line con 0
exec-timeout 0 0
\end{verbatim}

\subsection{Command Abbreviations}

\begin{tabular}{ | c | c | }
\hline
Up arrow or Control P & previous in command history \\ \hline
Down arrow or Control N & next in command history \\ \hline
Control A & go to beginning of current command \\ \hline
Control E & go to end of current command \\ \hline
Left arrow or Control B & back one character in current command \\ \hline
Right arrow or Control F & forward one character in current command \\ \hline
Backspace or Control D & delete one character in current command \\ \hline
ESC B & back one word in current command \\ \hline
ESC F & forward one word in current command \\ \hline
\end{tabular}

\subsection{History}

\texttt{show history} shows the last 10 commands that you ran at this level.\\

To set history size:

\begin{verbatim}
conf t
line con 0
history size 20
\end{verbatim}

\subsection{Major Commands}

Major commands switch you from one configuration mode to another (such as
interface configuration).
