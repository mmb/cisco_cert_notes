\section{Troubleshooting}

\subsection{Where To Begin}

Start at the physical layer of the OSI model. Is it turned on? Are you using
the right cable? Is the cable loose?\\

Crossover cables connect switches for trunking.\\

Rollover cables connect a laptop to the console port of a router.

\subsubsection{Cisco Discovery Protocol}

Cisco Discovery Protocol (CDP) can be used to check a Cisco device's physical
connections. It runs by default on Cisco routers and switches. It can be
enabled and disabled globally or per-interface.\\

To see if CDP is enabled:

\begin{verbatim}
show cdp
\end{verbatim}

To turn off CDP globally:

\begin{verbatim}
conf t
no cdp run
^Z
wr
\end{verbatim}

To turn on CDP globally:

\begin{verbatim}
conf t
cdp run
^Z
wr
\end{verbatim}

\subsubsection{CDP Neighbors}

\begin{verbatim}
show cdp neighbor
\end{verbatim}

Headers in CDP neighbors output:

\begin{description}

\item[device ID]
the hostname of the remote device

\item[local interface]
the interface on the local switch that is connected to the remote host

\item[holdtime]
number of seconds that the local device will retain the last CDP advertisement
received from the remote host

\item[capability]
what type of device the remote host is

\item[platform]
the remote device's hardware platform

\item[port ID]
the remote device's interface on the direct connection

\end{description}

To get more detail (including the IP addresses of remote hosts):

\begin{verbatim}
show cdp neightbor detail
\end{verbatim}

To control CDP per interface:

\begin{verbatim}
conf t
int fast 0/12
no cdp enable
\end{verbatim}

\subsection{show interface Output}

\begin{verbatim}
show int serial0
\end{verbatim}

\begin{verbatim}
Serial0 is administratively down, line protocol is down
\end{verbatim}

The interface is shut. Open it and wait a minute to make sure the line protocol
stays up.

\begin{verbatim}
int serial0
no shut
\end{verbatim}

\begin{verbatim}
Serial0 is down, line protocol is down
\end{verbatim}

Interface is physically down. Check cables.

\begin{verbatim}
FastEthernet0/3 is down, line protocol is down (err-disabled)
\end{verbatim}

Could be port security violation.

\begin{verbatim}
Serial0 is up, line protocol is down
\end{verbatim}

Physically ok, could be encapsulation mismatch or keepalives are not
being received.\\

To show the MAC address table:

\begin{verbatim}
show mac-address-table
show mac-address-table dynamic
\end{verbatim}

If you encounter a problem while configuring a router or switch, it's probably
because of something you just did.

\subsection{Troubleshooting with ping}

Cannot ping 172.23.23.3.

\begin{verbatim}
show int ethernet0  
\end{verbatim}

Check if physical interface is administratively down.\\

Check if IP address is configured on interface.

\begin{verbatim}
conf t
int e0
ip address 172.23.23.2 255.255.255.0
no shut
^Z
wr
\end{verbatim}

Check VLAN port membership. If hosts are on different VLANs they will not be
able to send IP packets to each other without a layer 3 device.

\begin{verbatim}
show vlan brief
\end{verbatim}

\subsection{PC Troubleshooting}

Show host network settings:

\begin{verbatim}
ipconfig /all
ping 127.0.0.1
ping default gateway
ping what you can't reach
\end{verbatim}

To release the PC's DHCP lease:

\begin{verbatim}
ipconfig /release
\end{verbatim}

To renew the PC's DHCP lease:

\begin{verbatim}
ipconfig /renew
\end{verbatim}

To display the host's routing table:

\begin{verbatim}
netstat -rn
\end{verbatim}

\subsection{WAN Troubleshooting}

Cisco routes must have a password set on the VTY lines to allow telnet or
ssh connections. Users must use an enable password or the VTY lines can be
set up to put the user into enable mode immediately.\\

telnet sends data in clear text. ssh encrypts all data but requires more
configuration and possibly extra hardware.\\

To show information about current telnet sessions:

\begin{verbatim}
show sessions
\end{verbatim}

To resume a telnet session by session id:

\begin{verbatim}
resume 3
\end{verbatim}

To suspend the current telnet session, hit Control-Shift-6 and then x.\\

To disconnection an open session by session id:

\begin{verbatim}
disconnect 3
\end{verbatim}

\subsubsection{ping}

Regular ping results:

\begin{description}

\item[!!!!!]
ip connectivity to the destination is ok

\item[.....]
no ip connectivity to destination

\item[U.U.U]
the local router has a route to the destination, but a downstream router does
not

\end{description}

\subsubsection{traceroute}

For identifying which downstream router in a path does not have a route to
the destination.

\begin{verbatim}
traceroute 208.109.62.234
\end{verbatim}

Control-Shift-6 is the escape sequence on Cisco routers. It will terminate
the currently running ping, traceroute, etc..\\

The Microsoft version of traceroute is called tracert.

\subsubsection{Extended ping}

You must be at the enable prompt to run extended ping.\\

Enter \texttt{ping} and it will prompt you for options.

\subsection{Route Administrative Distance}

When multiple routes exist to a destination, the router will choose the
longest match (for example a /26 route over a /24 route).\\

If the prefix lengths are all the same, the administrative distance is used.
The administrative distance is the first number in brackets in show ip route
output. The lower the administrative distance, the more believable the
route.\\

\begin{tabular}{ | c | c | }
\hline
Route Source & Administrative distance \\ \hline
directly connected &   0 \\ \hline
static             &   1 \\ \hline
EIGRP summary      &   5 \\ \hline
internal EIGRP     &  90 \\ \hline
IGRP               & 100 \\ \hline
OSPF               & 110 \\ \hline
ISIS               & 115 \\ \hline
RIP                & 120 \\ \hline
external EIGRP     & 170 \\ \hline
iBGP               & 200 \\ \hline
\end{tabular}

\subsubsection{Floating Static Routes}

Sometimes you might not want your static routes to have an administrative
distance of 1. For example, to give one a higher administrative distance than
RIP.

\begin{verbatim}
ip proute 1.1.1.1 255.255.255.255 172.12.23.23 121
\end{verbatim}

Troubleshooting is all about knowing the fundamentals and having a structured
approach.

\subsection{Media Issues}

\texttt{show interface} is the most useful command when troubleshooting media
issues.
