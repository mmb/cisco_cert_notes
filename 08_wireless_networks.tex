\section{Wireless Networking}

\subsection{Wireless Network Types}

Wireless networks are created on wireless access points (WAPs).\\

When connecting from a hotel or restaurant you are connecting to a lily pad
network. The WAPs in a lily pad network can be owned by different companies.
They create hotspots where Internet access is available and usually a username
and password is required.\\

The most tangible benefit of wireless networks is cost reduction.

\subsubsection{Ad-hoc Wireless Networks}

An ad-hoc wireless network is a wireless network with no WAP where the hosts
communicate directly. This is also known as an independent basic service set
or IBSS.

\subsubsection{Infrastructure WLANs}

A more common configuration is an infrastructure WLAN. In this configuration
all communication goes through the WAP. Clients will associate with the
access point that is received with the strongest signal.\\

There are two types of infrastructure WLAN:

\begin{description}

\item[BSS]
basic service set, has a single WAP

\item[ESS]
extended service set, has multiple WAPs

\end{description}

An ESS is the type of network with multiple cells that you would use when
travelling. 10 to 15 percent overlap is recommended beween extended service
areas.\\

A client must be able to shift data rates to communicate while moving.

\subsection{Standards and Ranges}

802.11 defines wi-fi overall.\\

\begin{tabular}{ | c | c | c | c | c | }
\hline
Standard & Typical data rate & Peak data rate & Indoor range & Operating frequency\\ \hline
802.11a & 25 Mbps  & 54 Mbps  & 100 ft. & 5 GHz \\ \hline
802.11b & 6.5 Mbps & 11 Mbps  & 100 ft. & 2.4 GHz \\ \hline
802.11g & 25 Mbps  & 54 Mbps  & 100 ft. & 2.4 GHz \\ \hline
802.11n & 200 Mbps & 540 Mbps & 160 ft. & 2.4 or 5 GHz \\ \hline
\end{tabular}\\

802.11b and 802.11g are compatible and many cards and WAPs that support one
also support the other.\\

Microwave ovens and some wireless telephones also share the 2.4 GHz band
and can cause interference. Solid objects can also cause problems.

\subsection{Infrared Wireless}

Infrared has a high data rate but a range that is too short to be practical.

\subsection{Spread Spectrum}

Spread spectrum is a method of spreading a signal over a range of frequencies.\\

Methods of spread spectrum:

\begin{description}

\item[FHSS]
frequency hopping spread spectrum, sender and receiver agree on the range of
frequencies to use

\item[DSSS]
direct sequence spread spectrum, spreads the signal over the entire range of
frequencies, 802.11 b, g, and n use this

\item[OFDM]
orthogonal frequency division multiplexing, splits the signal and sends the
signal fragments over different frequencies at the same time, 802.11a uses
this

\end{description}

Spreading the signal:

\begin{itemize}

\item increases the resistance to noise
\item allows the sharing of a frequency band
\item makes the signal more difficult to intercept

\end{itemize}

\subsection{Antenna Types}

\begin{description}

\item[Yagi]
or Yagi-Uda, sends its signal in a single direction, must always be aligned
correctly, also known as directional antennas and point-to-point antennas

\item[Omni]
sends its signal in all directions at once, also know as omnidirectional and
point-to-multipoint antennas

\end{description}

A Yagi antenna is helpful in bridging the distance between WAPs. Omni
antennas are used for connecting hosts to WAPs.

\subsection{CA vs CD}

Ethernet has CSMA/CD and wireless networking has CSMA/CA (carrier sense
multiple access with collision avoidance). CSMA/CA works much the same as
CSMA/CD:

\begin{itemize}

\item a host that wants to transmit listens to see if another host is
transmitting

\item if the channel is idle, the host invokes a random timer, when the timer
expires, the host listens again and then transmits

\item if the channel is busy, the host cannot transmit

\end{itemize}

As opposed to CSMA/CD, no jam signals are sent out. Collisions are avoided
and not detected.\\

Ethernet is capable of full duplex, but wireless networks are limited to half
duplex.

\subsection{SSIDs}

The SSID (service set identifier) is a name for a wireless network. If the
wireless client's SSID matches the AP SSID, communication can proceed. The SSID
is case-sensitive and can be up to 32 characters.\\

A laptop can be configured with a null SSID, resulting in the client asking
the AP for its SSID. If the AP is configured to broadcast its SSID,
communication can proceed.\\

If SSID broadcasting is disabled, the SSID must be set on the client.

\subsection{MAC Address Authentication}

Similar to Cisco switch port security, APs can keep a list of secure MACs
which are allowed to access the network.

\subsection{WEP, WPA and WPA2}

WEP came first. WPA evolved from WEP. WPA2 evolved from WPA.

\subsubsection{WEP}

WEP stands for wired equivalent privacy.

\begin{itemize}

\item clear-text keys
\item static keys (makes passwords easier to guess)
\item one-way authentication (client does not authenticate AP, making it easier
for rogue access points)
\item encryption scheme easily broken

\end{itemize}

WEP support two forms of authentication: open and shared key. When using open
authentication, any device can authenticate. If both devices are using WEP,
but the client key does not match the AP key, authentication will succeed but
data cannot be successfully passed.

\subsubsection{WPA}

WPA stands for wi-if protected access. WPA works with all wireless NICs but
may not be supported by old APs.

\begin{itemize}

\item two-way authentication, AP authenticates client and client authenticates
AP
\item dynamic keys and stronger encryption scheme using TKIP (temporal key
integrity protocol)
\item uses an 8-byte MIC (message integrity check) to protect against replay
attacks, spoofing and man-in-the-middle attacks
\item uses 802.1X or PSK (pre-shared keys) for authentication

\end{itemize}

TKIP makes it possible to use legacy hardware designed for WEP.\\

Both WEP and TKIP use the RC4 stream cipher, but TKIP protects RC4 keys using
per-packet key mixing, which results in every packet having a unique encryption
key.\\

WPA requires the use of a passphrase instead of a password. The recommended
length is 20 - 30 characters.\\

Potential issues with WPA:

\begin{itemize}

\item legacy issues
\item access points will shut down their BSS if they receive two packes in a
row with a bad MIC, vulnerable to denial-of-service attack
\item if a small passphrase is intercepted, an attacker can run a dictionary
attack to compromise the passphrase

\end{itemize}

\subsubsection{WPA2}

After WPA was ratified by the Wi-Fi Alliance, IEEE came out with 802.11i.
After 802.11i, the Wi-Fi Aliiance came out with WPA2. 802.11i and WPA2 are
fully compatible.\\

WPA2 uses AES-CCMP encryption while WPA uses TKIP/MIC.

\subsection{802.1X}

When using 802.1X, the access point encapsulates any 802.1X traffic that is
bound for the authentication server and sends it to the server.

\subsection{Wi-Fi Alliance}

The Wi-Fi Alliance is a global, nonprofit industry trade association that is
devoted to promoting the growth and acceptance of wireless LANs.
