\section{Introduction to Network Security}

\subsection{Firewalls}

A firewall is the basic protection against internet-based attackers. A
firewall is a physical device that filters packets heading for your network.
The filters are set up to block attempts to harm your network.\\

Firewalls can block traffic based on:

\begin{description}

\item[protocol]
don't allow IPX

\item[source IP address]
don't allow any traffic from 172.10.1.0/24 to exit the network

\item[port number]
don't allow any traffic sourced from port 23 to enter the network

\end{description}

If users can't receive email, make sure the incoming POP3 port (110) is open.\\

If users can't send email, make sure the outgoing SMTP port (25) is open.\\

If you need to block users from accessing the web, make sure the outgoing
HTTP port is blocked.

\subsection{Proxy Servers}

Proxy servers are used to cache web pages. They will request pages on behalf
of the original requester. Another common use for proxies is filtering
requests for specified web pages.\\

Proxy servers can serve as a kind of firewall because they perform packet
filtering.

\subsection{Denial of Service Attacks}

In a denial of service (DoS) attack, the attacker attempts to overwhelm a
server with TCP connection requests. Legitimate users will be unable to get to
the server due to network congestion. The attacked server's resources are so
busy it cannot answer legitimate requests or they crash under the load.\\

A DoS attack is designed more to damage the network than to steal data.

\subsection{Reconnaissance Attacks}

In a reconnaissance attack, the attacker uses different techniques to gather
information about your network's strengths and weaknesses. The intruder will
return later to use this information.

\subsection{Access Attacks}

In an access attack data is compromised or stolen for personal use, financial
gain, or to embarrass the target.

\subsection{Cisco Adaptive Security Appliance}

With so many threats, a firewall is not enough. The Cisco ASA Series controls
network and application traffic and delivers flexible virtual private network
(VPN) connectivity.\\

Cisco uses the term ``anti-x'' to describe all of the protections the ASA
offers (anti-virus, anti-spam, etc.).

\subsection{Cisco Intruder Detection System}

Cisco Intruder Detection System (IDS) provides complete intrusion protection.

\subsection{Cisco Intrustion Prevention System}

Cisco Intrusion Prevention System (IPS) is targeted more towards e-commerce.

\subsection{Preventing Virus Attacks}

Always keep your anti-virus program definitions up-to-date. Your network
must be protected from both internal and external virus threats. An
aggressive anti-virus strategy can greatly reduce your network's chance
of being hit by a virus.\\

A virus is a computer program that gets onto a computer without a user's
knowledge and performs an action that could be mischeivous or destructive.
Just like human viruses, they are spread by contact, frequently through
email.\\

A worm is a type of virus, but it can spread on its own without any help
from the infected host. It can replicate itself.\\

A trojan horse is a program installed by the user that looks legitimate
but has a hidden payload that attacks the user. This a common way that
keystroke loggers can get onto a user's computer.\\

Periodically remind users:

\begin{itemize}

\item don't open email attachments from anyone outside the company

\item don't download software and install it, especially ``fun and free''
software, such as games, animated cursors, etc.

\item don't open email from anyone you don't know

\end{itemize}

\subsubsection{Choosing and Configuring an Anti-virus Program}

Choose one that updates automatically.\\

The number one mistake that network admins and home PC owners make is not
keeping the program up-to-date. New viruses are created every day. Anti-virus
vendors monitor viruses and update virus definitions regularly.\\

Even with anti-virus programs, 100\% protection isn't guaranteed.

\subsubsection{Which Files Should be Scanned for Viruses}

\begin{itemize}
\item .com
\item .exe
\item .ocx
\item .dll
\item Microsoft Word Documents
\end{itemize}

If you turn your anti-virus program off temporarily, don't forget to turn it
back on.

\subsection{Locking Up Your Hardware}

Lock it up.

\subsection{Hiding a Sensitive Network}

VLAN's aren't just for limiting broadcast scope. They are also a security
feature. To hide hosts, put them in their own VLAN.
