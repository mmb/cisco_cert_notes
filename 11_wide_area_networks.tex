\section{Wide Area Networks}

WANs connect devices that are separated by wide geographic areas. They use the
services of carriers such as telephone companies, cable companies, satellite
systems, and network providers.\\

WANs use serial connections of various types to provide access to bandwidth.

\subsection{WAN Physical Devices}

\begin{description}

\item[CSU/DSU]
channel service unit / data service unit, service provider's half of the
communication, connects to local routers and tells the router the clock rate
(the speed at which it can send and receive data)

\item[demarcation point]
the point where responsibilty for the physical devices passes from the
network administrator to the service provider
\end{description}

The CSU/DSU and the cable connecting it to the router are the customer's
responsibility. The cabling on the other side of the CSU/DSU is the service
provider's responsibility.

\subsubsection{DCE and DTE}

The DCE (data communications equipment) provides the clock rate. The DTE
(data terminal equipment) receives the clock rate. By default a Cisco
router acts as the DTE.\\

You can connect two Cisco routers using a DTE/DCE cable connected to their
serial interfaces. One router serves as the DCE. This is a common setup
for home labs.

\begin{verbatim}
show controller serial 1  
\end{verbatim}

Shows no cable, DCE or DTE.

\begin{verbatim}
show int serial1
\end{verbatim}

First line of output refers to physical state, logical state.

\subsubsection{Setting Clock Rate on DCE}

\begin{verbatim}
conf t
int serial1
clock rate 56000
^Z
wr
show int s1
\end{verbatim}

When working with line protocols, it's not enough for the connection to
come up. It has to stay up.\\

\subsubsection{Troubleshooting Interfaces}

\begin{itemize}

\item if the interface shows as administratively down, it's simply shut
down and needs to be opened

\item if the interface shows as down, there is a physical problem (maybe
a loose cable)

\item if the interface is up but the line protocol is down, physically
all is well but there is a logical issue, most likely an encapsulation
mismatch or a keepalives are not being received

\end{itemize}

\subsection{Encapsulation Methods}

HDLC and PPP are the two data-link layer (layer 2) protocols used for
encapsulation on serial point-to-point links.\\

To change encapsulation on an interface:

\begin{verbatim}
conf t
int s1
encapsulation ppp
\end{verbatim}

\subsubsection{HDLC}

HDLC is the default method on Cisco routers. It specifies an encapsulation
method for data on synchronous serial data links using frame character and
checksum.\\

The version of HDLC that runs on Cisco routers is proprietary so it cannot
be used in environments with multiple vendors. Cisco HDLC adds a new type field
so multiple network layer protocols can share the same serial link.\\

HDLC includes support for both point-to-point and multipoint configurations.\\

HDLC includes a method for authentication.

\subsubsection{PPP}

PPP provides router-to-router and host-to-network connections over synchronous
and asynchronous circuits. It originally emerged as an encapsulation protocol
for transporting IP traffic over point-to-point links.\\

PPP allows for data compression to save bandwidth and HDLC does not.\\

PPP multilink allows multiple physical channels to be bundled into a single
logical channel. HDLC does not support multilink.\\

PPP allows the use of two authentication schemes (PAP and CHAP) and HDLC
does not support either.

\begin{verbatim}
ppp authentication chap pap
\end{verbatim}

means that if authentication fails using CHAP, then PAP will be attempted.\\

The LCP in PPP is used for establishment, configuration, and testing the
data-link connection.

\subsection{Circuit-switched Networks}

With circuit switching, a dedicated physical circuit is established, maintained,
and terminated through a carrier network for each communication session.\\

Circuit switching allows multiple sites to connect to the switched network of a
carrier and communicate with each other.\\

ISDN is an example of a circuit-switched network.

\subsection{Point-to-point Links}

A point-to-point (or serial) communication link provides a single,
preestablished WAN communications path from the customer premises through a
carrier network, such as a telephone company, to a remote network.\\

Carriers usually lease point-to-point lines, which is why point-to-point lines
are often called leased lines.\\

For a point-to-point line, the carrier dedicates fixed transport capacity and
facility hardware to the line of a customer.\\

Point-to-point links:

\begin{itemize}

\item require minimal expertise to install and maintain

\item usually offer a high quality of service

\item provide permanent, dedicated capacity that is always available

\end{itemize}

\subsection{An Introduction to Frame Relay}

Frame relay:

\begin{itemize}
\item is cheap
\item is reliable
\item implements no error or flow control
\end{itemize}

Ethernet interfaces on routers are only for LAN connections. Serial interfaces
are for WAN connections.\\

Frame relay providers guarantee a certain amount of bandwidth will be
available at any given time. This guaranteed bandwidth is referred to as
the CIR (committed information rate).\\

A frame relay cloud is a collection of a frame relay provider's switches.
All of the provider's switches are DCEs.\\

A path through a frame relay cloud is called a virtual circuit. The data
can take different paths through the cloud.\\

Frame relay is a packet-switching protocol. The packets may take different
paths and will be reassembled. Contrast with circuit-switching protocols
which have dedicated paths.\\

There are two types of virtual circuits. A PVC (permanent virtual circuit)
is available at all times and a SVC (switched virtual circuit) is only
available when certain criteria are met. PVC is more popular.

\subsection{Network Address Translation}

\begin{description}

\item[NAT]
network address translation  

\item[PAT]
port address translation

\end{description}

A common configuration is to have a pool of routable addresses and when a
packet from the LAN goes out to the WAN the router will assign one. The
router will maintain a NAT translation table that maps the routable addresses
to the non-routable addresses. This is how it knows which private IP to
send incoming packets on the WAN side to.\\

Many organizations don't have enough routable addresses to do this.\\

Port address translation (PAT) allows a network to use a single routable
address to provide multiple hosts with internet access. The address is the
same but the port number changes. It allows you to use the IP address that is
already configured on the router's serial interface.\\

The only device that knows NAT or PAT is being used is the router. It
is transparent to devices on the both the LAN side and the WAN side.\\

On the router there is no concept of PAT, both NAT and PAT are referred to
as NAT.

\begin{verbatim}
show ip nat translations
\end{verbatim}

\begin{description}

\item[inside global]
represents one or more inside local IP addresses to the outside world, routable
address

\item[inside local]
non-routable IP addresses configured on the hosts

\item[outside global]
IP address assigned to a host on the outside network, routable address

\item[outside local]
IP address of an outside host as it appears to the inside network, probably
a non-routable address

\end{description}

To clear a specific extended dynamic translation entry from the NAT translation
table:

\begin{verbatim}
clear ip nat translation inside
\end{verbatim}

\subsection{ATM}

ATM stands for asynchronous transfer mode. It does not handle frames like
frame relay, but places data into cells (cell-switching) each of which is 53
bytes (5 head and 48 data). It is implemented using virtual circuits.\\

ATM networks are faster than frame relay networks but are more expensive to
build and maintain. They require specialized hardware.

\subsection{Modems}

A modem can and sometimes does take the place of a CSU/DSU.\\

The word modem comes from modulation (digital to analog) and demodulation
(analog to digital).\\

Modems are not as fast as other methods and they tie up the phone line.

\subsection{DSL}

DSL technology is a circuit-switched connection technology that uses existing
twisted-pair telephone lines to transport high-bandwidth data, such as
multimedia and video, to service subscribers.\\

DSL technologies place upload (upstream) and download (downstream) data
transmissions at frequencies above this 4-kHz window, allowing both voice and
data transmissions to occur simultaneously on a DSL service.

\subsubsection{Types of DSL}

\begin{description}

\item[ADSL]
asymmetric DSL, works under the assumption that the user is going to download
more information than they upload, allows phone calls and internet access on
the same line at the same time, uses several different modulation methods, the
most common of which is G.lite (G.922.2) which requires no splitter at the
customer location.\\

G.lite is slower than true ADSL but still faster than dialup.

\item[IDSL]
ISDN sigital suscriber line, can upload and download at 144 kbps, uses ISDN
transmission coding, bundling together both ISDN channels and voice on one
circuit. 

\item[RADSL]
rate adaptive DSL, uses software to calculate the maximum download and upload
speeds and dynamically adjust the rates

\item[SDSL]
symmetric DSL, the sending and receiving speeds are the same, the phone cannot
be used at the same time

\item[VDSL]
very high bit-rate DSL, can deliver speed up to 52 Mbps but maximum distance is
4000 feet over copper

\end{description}

\subsection{Cable}

Cable modems enable two-way, high-speed data transmissions using the same
coaxial lines that transmit cable television.\\

Cable modem access provides speeds superior to leased lines, with lower costs
and simpler installation.\\

The original purpose of the Multimedia Cable Network System Partners Ltd. was
to define a product and system standard capable of providing data and future
services over CATV plants.

\subsection{Internet History}

U.S. Department of Defense researchers devised a way to break messages into
parts, sending each part separately to its destination, where reassembly of the
message would take place. Today, this method of data transmission is known as a
packet system.\\

In 1972, ARPANET developers created the first email messaging software to more
easily communicate and coordinate projects.\\

In 1984, DNS was introduced and gave the world domain suffixes (such as .edu,
.com, .gov, and .org) and a series of country codes.
