\section{Static Routing and RIP}

In the examples the last octet of the IP address for any physical interface
is the router number. For loopbacks, each octet is the router number.\\

In a network diagram if a line looks like it doesn't connect to anything
it's a loopback interface.\\

When you ping a remote address, you are sending five ICMP echo packets to
the destination.\\

When there are no static routes and no routing protocols running, you can
still talk to hosts on the same subnet.

\subsection{Adding a Loopback Interface}

\begin{verbatim}
conf t
int loopback 0
ip address 2.2.2.2 255.255.255.0
^Z
wr
show int loopback0
\end{verbatim}

\subsection{debug ip packet}

Never use this on a production network.

\begin{verbatim}
debug ip packet
ping 172.12.123.2
no debug ip packet
\end{verbatim}

\texttt{u all} or \texttt{undebug all} turns off all debugging. It's a safe
way to make sure you don't leave any debugs on when you are done.

\subsection{Static Routing}

The \texttt{ip route} command creates static routes. A static route can
be to a given host or network or it can be to a a default route (a route that
will be used when there is no other match in the routing table).

\begin{verbatim}
conf t
ip route 2.2.2.0 255.255.255.0 172.12.123.1
^Z
copy run start
show ip route
ping 2.2.2.2
\end{verbatim}

A route's destination can either be:

\begin{itemize}
\item the local router's exit interface (NOT the IP address)
\item the remote router's IP address
\end{itemize}

When you send pings it's not enough for the local router to have connectivity
to the remote network. The downstream routers also need connectivity to that
network.\\

\subsubsection{Default Routes}

\begin{verbatim}
conf t
ip route 0.0.0.0 0.0.0.0 172.12.123.2
^Z
copy run start
\end{verbatim}

The destination and network mask for a default static route are all 0's.
As with a normal static route the next hop can be an IP address or an
interface on the local router.\\

Default static routes serve two major purposes:

\begin{enumerate}
\item sending data to networks that have no specific entry in the routing
table
\item reducing the size of the routing table
\end{enumerate}

\subsubsection{Host Routes}

Host routes are routes that only match one host (255.255.255.255 mask).

\begin{verbatim}
ip route 2.2.2.2 255.255.255.255 172.12.123.1
\end{verbatim}

\subsubsection{Why Not to Use Static Routes}

Static routes have their place but are not very scalable. They do not
dynamically adapt to network changes and can be hard to manage as the network
grows.

\subsection{Dynamic Routing Protocols}

\begin{description}
\item[RIP]
Routing Information Protocol, a distance vector routing protocol
\item[IGRP]
Integrated Gateway Routing Protocol, a distance vector routing protocol
\item[EIGRP]
Enhanced Integrated Gateway Routing Protocol
\item[OSPF]
Open Shortest Path First, uses link-state advertisements (LSAs)
\end{description}

When a router receives a packet with a destination address that is within an
unknown subnetwork of a directly attached network, if the ip classless command
is not enabled, the packet will be dropped.

\subsubsection{Distance Vector Routing Protocols}

\begin{itemize}

\item RIP is an example

\item determines the direction (vector) and distance (hop count) to any network
in the internetwork

\item is also known as ``routing by rumor''.

\end{itemize}

\subsubsection{Link-State Routing Protocols}

\begin{itemize}

\item respond quickly to network changes

\item send periodic updates (link-state refreshes) at long time intervals,
approximately once every 30 minutes

\item every router tries to build its own internal map of the network topology

\end{itemize}

\subsubsection{Metrics}

Metrics most commonly used by routing protocols:

\begin{itemize}

\item hop count
\item bandwidth
\item delay

\end{itemize}

\subsubsection{RIP}

There are two version of RIP. Today RIPv2 is almost always used.\\

To remove static routes \texttt{no ip route ...}.\\

When a router is configured to run RIP, the router will send routing updates
out the interfaces that are RIP-enabled.

\begin{verbatim}
conf t
router rip
no auto-summary
network 172.12.123.0
network 1.0.0.0
^Z
show ip protocols
\end{verbatim}

By default RIP will send version 1 updates, but will accept any version.

\subsubsection{RIP Versions}

\begin{description}
\item[RIPv1]
classful, no VLSM, no manual route summarization
\item[RIPv2]
classless, VLSM support, manual route summarization support
\end{description}

To force RIP to send and receive only v2, use the \texttt{version 2}
command under the RIP process.

\begin{verbatim}
conf t
router rip
version 2
^Z
show ip protocols
\end{verbatim}

RIPv2 multicasts its updates to 224.0.0.9.\\

RIPv1 broadcasts its updates.\\

\texttt{show ip route rip} shows only routes discovered by RIP.\\

The two numbers in brackets in the output are [administrative distance/metric].\\

Directly connected routes show up as connected (C) instead of RIP (R) because
their administrative distance is lower (their existence is more believable
because the router knows about them firsthand).\\

To debug rip:

\begin{verbatim}
debug ip rip  
\end{verbatim}

\subsubsection{RIP Metrics}

The metric for RIP (versions 1 and 2) is hop count. Hop count is a measurement
of how far it is to to a remote destination. This is a serious limitation for
wide area networks (WANs) because RIP considers only hops and does not take
into account the speed of links.\\

\texttt{clear ip route *} clears all dynamic routes. Don't use it in
production.
