\documentclass{article}

\begin{document}

\begin{description}

\item[host]
network device with an IP address

\end{description}

\section{Converting Dotted Decimal to a Binary String}

Convert 192.168.1.100 to a binary string.\\

Each part is a decimal representation of a binary string (a string of 1's and
0's).\\

The remainder starts as the number being converted.\\

From left to right, if the column number is less than or equal to the
remainder, put a 1 in the column and subtract the column number from the
remainder.\\

\begin{tabular}{ | r | r | r | r | r | r | r | r | r | }
\hline
    & 128 & 64 & 32 & 16 & 8 & 4 & 2 & 1 \\ \hline
192 &   1 &  1 &  0 &  0 & 0 & 0 & 0 & 0 \\ \hline
168 &   1 &  0 &  1 &  0 & 1 & 0 & 0 & 0 \\ \hline
  1 &   0 &  0 &  0 &  0 & 0 & 0 & 0 & 1 \\ \hline
100 &   0 &  1 &  1 &  0 & 0 & 1 & 0 & 0 \\ \hline
\end{tabular}\\

\begin{tabular}{ | r | r | r | r | r | r | r | r | r | }
\hline
    & 128 & 64 & 32 & 16 & 8 & 4 & 2 & 1 \\ \hline
255 &   1 &  1 &  1 &  1 & 1 & 1 & 1 & 1 \\ \hline
255 &   1 &  1 &  1 &  1 & 1 & 1 & 1 & 1 \\ \hline
255 &   1 &  1 &  1 &  1 & 1 & 1 & 1 & 1 \\ \hline
  0 &   0 &  0 &  0 &  0 & 0 & 0 & 0 & 0 \\ \hline
\end{tabular}

\section{Subnetting}

The subnet mask tells which bits of the address are the network bits and
which are the host bits.\\

255.255.255.0 has 24 network bits and 8 host bits.\\

In 192.168.1.100 255.255.255.0, 192.168.1 is the network part.\\

Any IP addresses that have the same network portion are on the same subnet.\\

In the network is configured correctly, different subnets should be on different
``sides'' of the router.\\

Hosts on the same subnet should not be separated by a router.

\section{IP Address Classes}

RFC 791 defines IP address classes.

\begin{tabular}{ | c | c | c | p{1cm} | p{1cm} | p{4cm} | }
\hline
Class & First octet & Default netmask & Default network bits & Default host bits & Notes \\ \hline
Class A & 1 - 126 & 255.0.0.0 & 8 & 24 & \\ \hline
Class B & 128 - 191 & 255.255.0.0 & 16 & 16 & \\ \hline
Class C & 192 - 223 & 255.255.255.0 & 24 & 8 & \\ \hline
Class D & 224 - 239 & & & & reserved for multicasting \\ \hline
Class E & 240 - 255 & & & & reserved for future use, experimental \\ \hline
\end{tabular}\\

Any address with a first octet of 127 is reserved for loopback interfaces
(but not for Cisco router loopback interfaces).\\

Know:

\begin{itemize}

\item how to identify the class of an address
\item which addresses can be assigned to hosts (A, B, C)
\item which addresses cannot be assigned to hosts (D, E, 127)
\item default network mask, network bits and host bits for A, B, C

\end{itemize}

\section{Private IP Address Ranges}

RFC 1918 defines private address ranges. They are used for internal networks.\\

\begin{tabular}{ | c | c | c | c | }
\hline
Class & Private address range & Private address netmask & Private address prefix \\ \hline
Class A & 10.0.0.0 - 10.255.255.255 & 10.0.0.0 255.0.0.0 & 10.0.0.0/8 \\ \hline
Class B & 172.16.0.0 - 172.31.255.255 & 172.16.0.0 255.240.0.0 & 172.168.0.0/12 \\ \hline
Class C & 192.168.0.0 - 192.168.255.255 & 192.168.0.0 255.255.0.0 & 192.168.0.0/16 \\ \hline
\end{tabular}

\section{Routing}

When a PC wants to send data, there are two possibilites: the destination
is on the same network or it's on another network.\\

If they are on the same network, the router does not need to get involved.\\

If the sending hosts knows it's not on same network as the destination, it will
send the packets to its default gateway.\\

When a router receives a packet there are 3 possibilities:

\begin{enumerate}

\item destined for a directly connected network
\item destined for a non-directly connected network that the router has an
entry for in its routing table
\item destined for a non-directly connected network that the router does not
have an entry for in its routing table

\end{enumerate}

\texttt{show ip route} shows the routes on a router\\

Static routes are create with the ip route command. For example:\\

\texttt{ip route 30.0.0.0 255.0.0.0 ethernet1}\\

\end{document}
