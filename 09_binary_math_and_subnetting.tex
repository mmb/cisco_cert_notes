\section{Binary Math and Subnetting}

\subsection{The Secret of Binary Success}

It is easier to learn binary math than to try to learn tricks for binary math.

\subsection{Decimal to Binary}

Convert 231 and 123 to binary strings.\\

The remainder starts as the number being converted.\\

From left to right, if the column number is less than or equal to the
remainder, put a 1 in the column and subtract the column number from the
remainder.\\

\begin{tabular}{ | r | r | r | r | r | r | r | r | r | }
\hline
    & 128 & 64 & 32 & 16 & 8 & 4 & 2 & 1 \\ \hline
231 &   1 &  1 &  1 &  0 & 0 & 1 & 1 & 1 \\ \hline
123 &   0 &  1 &  1 &  1 & 1 & 0 & 1 & 1 \\ \hline
\end{tabular}

\subsection{Binary to Decimal}

Convert 10101010 to decimal.\\

Add the columns that contain a 1.\\

\begin{tabular}{ | r | r | r | r | r | r | r | r | }
\hline
128 & 64 & 32 & 16 & 8 & 4 & 2 & 1 \\ \hline
  1 &  0 &  1 &  0 & 1 & 0 & 1 & 0 \\ \hline
\end{tabular}

\begin{equation}
128 + 32 + 8 + 2 = 170
\end{equation}

\subsection{Subnetting Basics}

\begin{description}
  
\item[Class A]
1 - 126, 8 network bits, 24 host bits, default mask 255.0.0.0
\item[Class B]
128 - 191, 16 network bits, 16 host bits, default mask 255.255.0.0
\item[Class C]
192 - 223, 24 network bits, 8 host bits, default mask 255.255.255.0

\end{description}

17.1.1.1 is class A so it has 8 network bits and 24 host bits. The network
part is 17 and the host part is 1.1.1.\\

Subnetting is borrowing host bits. It conserves addresses by allowing the
use of smaller sized network blocks.\\

The practical minimum number of bits that can be borrowed to form a subnet
is two.

\subsection{Calculating the Number of Valid Subnets}

The number of valid subnets is:

\begin{equation}
2 ^ {\mbox{ number of subnet bits}}
\end{equation}

If you see 'no ip subnet-zero' near the top of the configuration, you need to
subtract 2 to get the correct answer. That means the routing protocol is
classful (RIPv1 or IGRP).\\

If a classless protocol is being used (RIPv2, EIGRP, OSPF) you should not
subtract 2. In this case 'ip subnet-zero' will appear in the configuration
(this is the default setting).\\

The term VLSM (variable-length subnet masking) is used.\\

If no additional information is given, you should not subtract 2.\\

Example:\\

Network is 172.20.0.0, subnet mask is 255.255.255.0 (/24)\\

It's a class B, so the default mask is 255.255.0.0.

\begin{verbatim}
class B default mask 11111111 11111111 00000000 00000000
                 /24 11111111 11111111 11111111 00000000
\end{verbatim}

The bits where the default mask and subnet mask both have 0's are the host
bits.\\

The bits where they share 1's are the network bits.\\

The bits where the default mask has 0's but the subnet mask has 1's are the
subnet bits (the bits that are borrowed from the host bits).\\

There are 8 subnet bits, so there are:

\begin{equation}
2 ^ 8 = 256 \mbox{ valid subnets}
\end{equation}

\subsection{Prefix Notation}

Prefix notation is a slash followed by the number of 1's in the mask.\\

/24 means 24 1's at the beginning of the mask.

\subsection{Calculating Number of Valid Host Addresses in a Subnet}

The number of valid hosts on a subnet is:

\begin{equation}
2 ^ {\mbox{ number of host bits}} - 2
\end{equation}

Example:\\

150.50.50.0 /24\\

150.50.50.0 is a class B, so the default network mask is /16.

\begin{verbatim}
class B default mask 11111111 11111111 00000000 00000000
                 /24 11111111 11111111 11111111 00000000
\end{verbatim}

There are 8 host bits, so

\begin{equation}
2 ^ 8 - 2 = 254 \mbox{ valid hosts}
\end{equation}

The two hosts that are subtracted are the network address (the first address in
the range) and the broadcast address (the last address in the range).

The broadcast address, also known as the directed broadcast address, is an IP
address that specifies all hosts on a specified network. It has all ones on the
host field.

\subsection{Calculating the Subnet Number of a Given IP Address}

Convert both the IP address and its subnet mask to binary and boolean AND
them together (if both bits are 1, the result is 1, otherwise the result is
0).\\

Examples:

\begin{verbatim}
178.56.21.9 10110010 00111000 00010101 00001001
        /24 11111111 11111111 11111111 00000000
            10110010 00111000 00010101 00000000 = 178.56.21.0 /24
\end{verbatim}

\begin{verbatim}
200.154.150.89 11001000 10011010 10010110 01011001
           /27 11111111 11111111 11111111 11100000
               11001000 10011010 10010110 01000000 = 200.154.150.64 /27
\end{verbatim}

\subsection{Determining the Range of Valid Host Addresses on a Subnet}

To determine the range of valid host addresses in a subnet, first determine
how many overall host addresses are in that subnet. The first address is the
network number and is not a valid host address. The last address is the
broadcast address and is not a valid host address. All the addresses in
between are valid host addresses.\\

Example:

\begin{verbatim}
200.154.150.89 11001000 10011010 10010110 01011001
           /27 11111111 11111111 11111111 11100000
200.154.150.64 11001000 10011010 10010110 01000000
\end{verbatim}

There are 5 host bits. The first address in the range is 200.154.150.64.\\

To get the last address, calculate the value with all host bits were set to 1.\\

11001000 10011010 10010110 01011111 = 200.154.150.95\\

The valid host addresses are 200.154.150.65 - 200.154.150.94.

\subsection{Meeting Stated Design Requirements}

Your network uses Class B network 165.10.0.0. We need at least 150 subnets
that have no more than 200 hosts each. Which subnet mask should you use?\\

\begin{equation}
\mbox{number of subnets} = 2 ^{\mbox{ number of subnet bits}}
\end{equation}

\begin{equation}
\mbox{number of valid hosts} = 2 ^{\mbox{ number of host bits}} - 2
\end{equation}

165.10.0.0 is a class B.

\begin{verbatim}
165.10.0.0           10100101 00001010 00000000 00000000
default network mask 11111111 11111111 00000000 00000000 (/16)
\end{verbatim}

9 bits for the network part gives 512 subnets, 126 valid hosts per subnet.\\

The subnet mask should be 255.255.255.128 (/25).
